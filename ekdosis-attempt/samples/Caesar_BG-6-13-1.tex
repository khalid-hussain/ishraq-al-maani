\documentclass[12pt]{article}
\usepackage{fontspec}
\usepackage[latin.classic,french,english]{babel}
\babelfont{rm}{Old Standard}
\babelfont{sf}{NewComputerModernSans10}
\babelfont{tt}{NewComputerModernMono10}

\usepackage{nextpage}
\usepackage{xltabular}

\usepackage[teiexport=tidy]{ekdosis}
\DeclareApparatus{default}[
        delim=\hskip0.75em,
        ehook=.]

\FormatDiv{2}{}{.}

\SetAlignment{
  tcols=3,
  lcols=1,
  texts=latin[xml:lang="la"];
        english[xml:lang="en"];
        french[xml:lang="fr"],
  apparatus=latin,
  segmentation=auto}

\AtBeginEnvironment{latin}{\selectlanguage{latin}}
\AtBeginEnvironment{english}{\sloppy\selectlanguage{english}}
\AtBeginEnvironment{french}{\sloppy\selectlanguage{french}}

\DeclareWitness{A}{A}{\emph{Bongarsianus} 81}[
               msName=Bongarsianus,
               settlement=Amsterdam,
               idno=81,
               institution=University Library,
               origDate=s. IX--X]
\DeclareHand{A1}{A}{A\textsuperscript{1}}[\emph{Emendationes
               scribae ipsius}]
\DeclareWitness{M}{M}{\emph{Parisinus Lat.} 5056}[
               origDate={s. XII}]
\DeclareWitness{B}{B}{\emph{Parisinus Lat.} 5763}[
               origDate={s. IX--X}]
\DeclareWitness{R}{R}{\emph{Vaticanus Lat.} 3864}[
               origDate={s. X}]
\DeclareWitness{S}{S}{\emph{Laurentianus} R 33}[
               origDate={s. X}]
\DeclareWitness{L}{L}{\emph{Londinensis} Br. Mus. 10084}[
               origDate={s. XI}]
\DeclareWitness{N}{N}{\emph{Neapolitanus} IV, c. 11}[
               origDate={s. XII}]
\DeclareWitness{T}{T}{\emph{Parisinus Lat.} 5764}[
               origDate={s. XI}]
\DeclareWitness{f}{\emph{f}}{\emph{Vindobonensis} 95}[
               origDate={s. XII}]
\DeclareWitness{U}{U}{\emph{Vaticanus Lat.} 3324}[
               origDate={s. XI}]
\DeclareWitness{l}{\emph{l}}{\emph{Laurentianus} Riccard. 541}[
               origDate={s. XI--XII}]
\DeclareShorthand{a}{α}{A,M,B,R,S,L,N}
\DeclareShorthand{b}{β}{T,f,U,l}

\begin{document}

\begin{xltabular}[c]{0.75\linewidth}{lXl}
  \caption*{\textbf{Conspectus siglorum}\label{tab:conspectus-siglorum}}\\
  \multicolumn{3}{c}{\emph{Familia} \getsiglum{a}}\\
  \SigLine{A}\\
  & \getsiglum{A1} \emph{Emendationes scribae ipsius} & \\
  \SigLine{M}\\
  \SigLine{B}\\
  \SigLine{R}\\
  \SigLine{S}\\
  \SigLine{L}\\
  \SigLine{N}\\
  \multicolumn{3}{c}{\emph{Familia} \getsiglum{b}}\\
  \SigLine{T}\\
  \SigLine{f}\\
  \SigLine{U}\\
  \SigLine{l}\\
\end{xltabular}

\cleartoevenpage

\begin{alignment}
  \begin{latin}
    \ekddiv{head=XIII, depth=2, n=6.13, type=section}
    In omni Gallia eorum hominum qui \app{
      \lem[wit=a]{aliquo}
      \rdg[wit=b, alt=in al-]{in aliquo}}
    sunt numero atque honore genera sunt duo. Nam plebes paene
    seruorum habetur loco, quae \app{
      \lem[wit={A,M}, alt={nihil audet (aut et \getsiglum{A1})
        per se}]{nihil audet per se}
      \rdg[wit=A1,nordg]{nihil aut et per se}
      \rdg[wit={R,S,L,N}]{nihil habet per se}
      \rdg[wit=b]{per se nihil audet}}, \app{
      \lem[wit=a]{nullo}
      \rdg[wit=b]{nulli}} adhibetur \app{
      \lem{consilio}
      \rdg[wit={T, U}, alt=conc-]{concilio}}.
  \end{latin}
  \begin{english}
    \ekddiv{head=XIII, depth=2, n=6.13, type=section}
    Throughout all Gaul there are two orders of those men who are of
    any rank and dignity: for the commonality is held almost in the
    condition of slaves, and dares to undertake nothing of itself,
    and is admitted to no deliberation.
  \end{english}
  \begin{french}
    \ekddiv{head=XIII, depth=2, n=6.13, type=section}
    Partout en Gaule il y a deux classes d'hommes qui comptent et qui
    sont considérés. Quant aux gens du peuple, ils ne sont guère
    traités autrement que des esclaves, ne pouvant se permettre aucune
    initiative, n'étant consultés sur rien.
  \end{french}
\end{alignment}

\end{document}
